\subsubsection{}
Buď L jazyk predikátové logiky 1. řádu a rovností, jedním binárním predikátovým
symbolem $p$ a jedním unárním funkčním symbolem $f$.  Nechť $T$ je teorie 1.
řádu s jazykem L daná následujícími dvěma speciálními axiomy:

$$ p(f(x), x)$$
$$f(f(x)) = f(f(y)) \Rightarrow p(x,y)$$

Uvažujme realizaci $M=(\mathbb{Q}, \leq, h)$ jazkyka L, kde $\leq p_{M}$ a
operace $h=f_{M}$ na množině $\mathbb{Q}$ je definována předpisem $h(a) =
\frac{a}{2}$ pro libovolné $a \in \mathbb{Q}$. Rozhodněte, zda:

\begin{enumerate}[a)]
  \item $M$ je modelem teorie $T$
  \item $f(f(x)) = f(f(y)) \Rightarrow p(f(x), y)$ je důsledkem teorie T.
\end{enumerate}
