Nechť L je jazyk s rovností, bin. predikatovy sym. p a binarnim funkcnim symbolem
p. Uvazujte formule 
$$\phi \equiv f(x,x)=x$$
$$\chi \equiv p(x,f(x,y)$$
$$\psi \equiv p(x,y) \Leftrightarrow f(x,y)=y$$
$$\omega \equiv f(x,f(y,z))=f(f(x,y),z)$$
a teori $T = {\phi, \chi, \psi, \omega}$.

\begin{enumerate}[a)]
  \item Uvazujte realizaci R jaz. do L s univerzum N a realizaci symbolu
  $p_R(a,b)\Leftrightarrow a \mid b$ ($\mid$ znamena deli) a $f_R = nsn(a,b)$ rozhodnut
  zda $R \models T$
  \item Zjistite zda plati $T\setminus\{\chi\}\cup\{\omega\}\models\chi$ a
  zduvodnete
\end{enumerate}
